\documentclass[a4paper,10pt]{article}

\usepackage{fullpage}
\usepackage{fancyhdr}
\usepackage{lastpage}
\usepackage{amsmath}
\usepackage{graphicx}
\DeclareGraphicsExtensions{.pdf,.png,.jpg}



\begin{document}


\title{Concurrency in Go}
\maketitle

\subsection*{Objective:}
Your assignment will be to make a 
sequential Symbolic Regression engine
run in parallel using Go's built in concurrency features.
There will be two tasks:
(1) syncronous execution of the sub-search processes and
(2) asyncronous communication between the sub-search processes. 
You will only be concerned with data flow through the
search process. The low-level details have been taken care of for you.

\noindent
All questions can be directed to Tony Worm (verdverm@gmail.com)

\subsection*{Background:}
Symbolic Regression (SR) is a generalization of (non)linear regression.
SR searches for the form of a model when the equation is unknown.
The standard implementation is an evolutionary algorithm known as Genetic Programming.
Equations are represented as an AST and are evaluated, selected, and breed
much like Darwin's model of evolution.

A single search process will initially create a pool of equations,
known as a population. The entire population is evaluated on the
training data, producing a per-equation error value.
The population is then sorted with the Pareto non-dominated sort,
which balances accuracy(error) and complexity(size).
A new population of children is then created by selecting
two parents psuedo-randomly, crossing subtrees of their ASTs,
and possibly mutating each offspring.
The offspring are then evaluated to determine their fitness.
During normal iteration, the parent and offspring populations
are combined and sorted together. This Allows offspring to 
replace parents, but also permits good parents to survive in
the population. This process is repeated for a number of iterations,
also known as generations, at which point a set of equations
are returned. The returned set provides a range of equations
from simple to complex with increasing accuracy.
This is what island.go does.

To improve the consistancy of the SR algorithm,
several of the searches are run simultaneously.
Each individual search is known as an Island
and are coordinated by a master Search 
process (search.go). Your assignment
is to use Go's concurrency features
to manage the flow of information
between the Search process
and the Island processes.

\subsection*{Task 1: Installation}
Download and install the Go software (golang.org).
The home page has a download link and instructions.
If you are using linux, many distros have the Go language in the repositories now,
so you can check there too.
There are also many good resources for learning Go's syntax on the website.
See the "What's next" section on the installation page.

\noindent
The homework code is located at github.com/verdverm/go-eureqa.
You should be able to \textbf{go get github.com/verdverm/go-eureqa}
to obtain the homework code. It should be located in the Go
working directory setup during the installation process.
You may also download a zip or use git if you wish.

Once you have the code, navigate to the directory and run
\textbf{go build}. You can then run the program by
\textbf{./go-eureqa}. (Windows may have alternate syntax for running the program)

Four data sets are provided. Use the -data="filename" argument
to the go-eureqa program. 
(You don't need to include the data/ dir when supplying the filename)

You should only need to make modifications to three files.
\begin{itemize}
\item main.go: sets up parameters, initializes a search, and calls the search run function.
\item search.go: the master search process sets up the islands and cooridates the search process.
\item island.go: implements the SR algorithm
\end{itemize}

The other files are helpers. data.go handles reading the data in and is used by the islands.
eqn\_funcs.go are implementations of the SR algorithms individual functions.
The go-symexpr (which you may not see in the local directory) is a library
which implements symbolic expressions. This should automatically be
downloaded and installed when you \textbf{go build} the assignment.

\subsection*{Task 2: Running in parallel}
Currently, the islands are run sequentially, in a loop,
in the Search.runSearch() function.
In order to make the islands run concurrently,
you will need to launch them as gorountines
using the \textbf{go} keyword.

Modify the Island.initIsland() function to
return a channel on which commands can be
sent to the island from the main search process.
Add a slice of \textbf{chan int} to the Search struct
for saving each islands returned channel.
The commands that should be implemented are:
\textbf{start,stop, and quit}.
You can assign a unique integer to each command
or use Go's const to give them names.
The Island goroutines can be launched in the Search.initSearch()
function, but they should not start processing until
they have received the start command.

You will need to add a function to island.go called
\textbf{runIsland}. This function will use a 
select statement on the command channel
and run Island.step() in the default case,
but only if the start command has been received.
Receiving the stop command should cause
the island to stop processing, but should
not cause the goroutine to exit. This is
what the quit command is for, and should
be a round trip communication.

Finally, you will need to modify the Search.runSearch()
function. There should be a loop at the beginning
which sends the start command to each island.
The current, first loop should be changed so
it only receives results from the islands.
This will provide the synchronization
of the islands so that they stay at the same iteration.
After the maximum number of iterations
is reached, each island should be sent
the stop command. Once all islands
have stopped processing, they should
send a final report. Once the search
process receives the final reports,
it should send the quit command
to each island and wait for the
return message before exiting.


\subsection*{Task 3: Adding migration}
Migration is the term used in the SR field
to describe the sharing of current results
between islands. It allows each island
to share what it has learned thus far 
in its searching with its neighbors.
This allows information to flow 
between islands, creating a coordinated search.
You will implement the sharing of 
current best equations using buffered channels.
The communication topology will be a ring, where
each Island is connected to two neighbors.

The setup will take place in Search.initSearch()
and the operation will take place in island.go

Add two parameters to the SR\_Params struct:
(1) NumMigrants, the number of equations to send on the channel.
(2) MigrationRate, the number of iterations between migrations.
In general, we want to send a few equations to our neighbors
every few iterations.

Add a slice of \textbf{chan []*Eqn}to the Search struct to hold the 
will need to have channels its two neightbors.
Look at the reporting mechanisms for an example
of a channel that sends a slice of equations.
Two islands will share a channel
and migration should be bi-directional.
(hint: there is a simple loop and modulous pattern for matching islands and channels)

Implementing the migration mechanisms for each island
will require several small changes.
\begin{itemize}
\item a slice in the Island struct for holding the incoming migrants.
\item a function for receiving the migrants and storing them.
\item a function or addition to another function for for sending migrants.
\item modifying the Island.selectEqns() function to include the migrants in the selection step.
\end{itemize}


\subsection*{Task 4: Writeup}
Create a document describing the changes you made,
referencing file and line numbers.

You should also report the best 8 equations
returned for each data set.

\textbf{Is this needed? If so, what else would you like here?}

\subsection*{If you want to:}
The current SR implementation is very basic,
and likely incapable of finding the equations
in the provided data sets.
Here are some topics for enhancements.
\textit{These enhancements are not required for the assignment.}
They will, however, vastly improve the effectiveness of the algorithm.
Make sure you have completed the assignment befor
\begin{itemize}
\item Asyncronous reporting
\item Co-evolution of fitness predictors
\item Age-layered populations
\item Brood selection
\end{itemize}


\end{document}

